% \usepackage[letterpaper,right=1.5in,top=0.8in,bottom=0.8in,left=0.8in]{geometry}
\usepackage{hyperref}
\usepackage{tocloft}
\usepackage{commath}
\usepackage{amsmath}
\usepackage{amssymb}
\usepackage{amsfonts}
\usepackage{amsthm}
\usepackage{mathtools}
\usepackage{mathabx}
\usepackage{esdiff}
\usepackage{caption}
\usepackage{esvect}
\usepackage{makeidx}
\usepackage{xcolor}
\usepackage[most]{tcolorbox}
\usepackage{enumitem}
\usepackage{newclude}
\usepackage{siunitx}
\usepackage{listings}

\usepackage{physics}
\usepackage{fullpage}

% Bibliography
\usepackage[style=verbose-ibid,backend=biber]{biblatex}
\addbibresource{references.bib}

% Color hyperlinks
\hypersetup{
    colorlinks,
    linkcolor = blue,
    citecolor = blue
}

% Number by section
\numberwithin{equation}{section}
\numberwithin{figure}{section}

% % LaTeX can vectors
\newcommand{\ora}[1]{\overrightarrow{#1}}
\DeclareRobustCommand*{\ora}{\overrightarrow}
\newcommand{\unitvec}[1]{\mathbf{e}_{\mathbf{#1}}}
\newcommand{\vecpart}[1]{\left\langle #1 \right\rangle}
\renewcommand{\v}{\mathbf}

% % LaTeX can norms
\newcommand{\veclen}[1]{\left\lvert#1\right\rvert}
\let\abs\veclen

% Overbar used for repetition
\newcommand{\overbar}[1]{\mkern 1.5mu\overline{\mkern-1.5mu#1\mkern-1.5mu}\mkern 1.5mu}

% amsthm styles
\theoremstyle{definition}\newtheorem{defn}{Definition}[section]
\newtheorem{mathrule}{Rule}[section]
\theoremstyle{plain}\newtheorem{thm}{Theorem}[section]
\newtheorem{cor}{Corollary}[thm]
\theoremstyle{definition}\newtheorem{ex}{Example}[section]
\theoremstyle{definition}\newtheorem*{soln}{Solution}

% autoref names
\def\sectionautorefname{Section}
\def\tableautorefname{Table}
\def\defnautorefname{Definition}
\def\thmautorefname{Theorem}
\def\corautorefname{Corollary}

% LaTeX can derivatives and also st
\newcommand{\ddx}[1]{\diff{}{x}\left[#1\right]}
\newcommand{\ddt}[1]{\diff{}{t}\left[#1\right]}
\newcommand{\st}{\text{ st. }}
\newcommand{\setst}{\ \middle\rvert\ }

% Custom math operators
\DeclareMathOperator{\qnorm}{N}

% Only number a specific equation in eg. an align environment
\newcommand\numberthis{\addtocounter{equation}{1}\tag{\theequation}}
